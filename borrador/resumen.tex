%\begin{abstract}
El contenido del presente documento representa la entrega parcial, para el periodo 2012-02, de la Memoria del proyecto “Sistema experto de soporte en el planeamiento estratégico militar”. En este trabajo, se aplica la metodología CommonKADS para el desarrollo de la gestión de conocimiento de un sistema basado en conocimiento, que está orientado al soporte del planeamiento estratégico militar, por medio de una plataforma web.

A continuación se describe el contenido del documento:
El capítulo 1 de esta memoria presenta el marco teórico del proyecto. En esta parte, se expondrán los fundamentos de las operaciones tácticas militares, la metodología CommonKADS1 y los sistemas basados en conocimiento (SBC).
En el capítulo 2 se describe el proyecto. En este acápite, se detallará el problema, los objetivos, indicadores de éxito, alcance y beneficios. Por otro lado, también se especifica la organización del proyecto y los riesgos del mismo.
En el capítulo 3 se muestra el desarrollo de los modelos contextuales CommonKADS. Entre estos se encuentran; el modelo organizacional, donde se estudia la organización para determinar el alcance del proyecto y conocer el entorno en el que se implantará; el modelo de tareas, donde se describen las características de las tareas involucradas en el proceso en el que se implantará el proyecto; y, por último, el modelo de agentes, donde se describen los agentes implicados en las tareas. 
El capítulo 4 contiene el desarrollo de los modelos conceptuales CommonKADS. Entre estos se encuentran el modelo de comunicación, que es donde se detalla el intercambio de información entre agentes, y el modelo de conocimiento, que es donde se modela el conocimiento que usan los agentes para la resolución de problemas.
Finalizando con la metodología CommonKADS, en el capítulo 5 se abarca todo lo concerniente con el modelo de diseño del software. Además, se explica la arquitectura del sistema de forma detallada.
En el capítulo 6, se listan los requerimientos solicitados por el cliente a través de las historias de usuario.
Por último, en el capítulo  7 se explica cómo se ha ido desarrollando el proyecto a lo largo de sus iteraciones.
Seguidamente, se encuentran los anexos a los que se hace referencia dentro del documento. 
El final de documento expone la bibliografía que se utilizó como ayuda para el desarrollo del sistema.
%\end{abstract}
