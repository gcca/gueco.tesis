\chapter{Modelo Conceptual}


%% >>> : - . - . - . - . - o - . - . - . - . - :

\section{Modelo de Conocimiento}

En esta sección se especifican los requerimientos sobre el conocimiento y los procesos de razonamiento que un experto aplica para la resolución de un problema. El \emph{modelo de conocimiento} distingue entre el conocimiento de la aplicación y el conocimiento de resolución del problema. El conocimiento de la aplicación se divide en tres subniveles:\todo{FALTA, todo}

\begin{description}
\item[Nivel del dominio] Conocimiento declarativo sobre el dominio, compuesto por los factores\footnote{Manejo de escenarios, }.
\item[Nivel de inferencia] Biblioteca de estructuras genéricas de inferencia, para la integración con \gls{clips}.
\item[Nivel de tarea] (orden de las inferencias).
\end{description}

Este modelo describe el conocimiento que tiene un determinado agente y que es relevante para la realización de una determinada tarea, además de describir la estructura del mismo en función de su uso. 

Los aspectos que definen las reglas vienen dados por el terreno, la meteorología y el enemigo de la misión. Con el diseño de la base de conocimiento como soporte para las inferencias, se procederá a analizar el escenario mediante un proceso deductivo.
Así por ejemplo, se tienen reglas como “Si existen precipitaciones intensas, con temperatura tropical en una estrategia ofensiva, entones hay un efecto negativo en el armamento enemigo”.
Para el diseño de la base de conocimiento, de las reglas y hechos que la componen, se usarán los siguientes aspectos, los cuales fueron proporcionados por el experto.
