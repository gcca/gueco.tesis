\chapter{Modelo Contextual}

%% >>> : - . - . - . - . - o - . - . - . - . - :

\section{Modelo de la organización}


%% >>>
\subsection{OM-1: Problemas y oportunidades}
\todo{Cambiar-enfocar estrategia por operaciones o maniobra}.

\begin{longtable}{p{.45\textwidth}|p{.48\textwidth}}

\textbf{Problemas} & \textbf{Oportunidades} \\\toprule
\endhead


Los expertos humanos sufren de agotamiento físico, haciendo que se retrasen en el diseño de sus operaciones.
&
Un sistema informático no sufre agotamiento físico.
\\\midrule


Al analizar un escenario complejo el experto necesita más tiempo para el análisis y podría cometer un error al realizar los cálculos o analizar la situación.
&
Un sistema experto sigue estrictamente un proceso de inferencia, sin errar en la aplicación del conocimiento.
\\\midrule


El especialista podría no estar disponible por enfermedad u otras labores encargadas.
&
El sistema informático podría estar desplegado en más de una instancia, con redundancia de datos, ofreciendo alta disponibilidad de la base de conocimiento.
\\\midrule


El conocimiento es incompleto, debido a que el experto solo domina un tipo de estrategia y parte del conocimiento es obtenido empíricamente.
&
La base de conocimiento puede ser elaborada por muchos expertos.
\\\midrule


Cuando el plan establecido para una misión varía el conjunto operacional tarda en enterarse de esos cambios.
&
El sistema puede realizar el envío a distintos dispositivos.
\\\bottomrule

\caption{Problemas y oportunidades}
\label{cuadro:om1problemasyoportunidades}

\end{longtable}
