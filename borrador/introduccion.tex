\chapter{Introducción}

A \todo{Volver a escribir esto: no cuadra con una tesis}lo largo de la historia mundial, las estrategias militares han estado presentes y en muchas ocasiones han sido un factor primordial en el rumbo de los pueblos. Por ejemplo, en la Batalla de Poitiers el ejército de Carlos Martel estableció una defensa sólida que les proporcionó la victoria, a pesar de contar con inferioridad numérica de soldados, obligando a los musulmanes invasores a retirarse y preservando el cristianismo como la fe dominante en Europa\footnote{Cfr. \citealt{Calliope21}.}. En el Perú, la operación Chavín de Huántar, llevada a cabo el 22 de Abril de 1997, permitió el rescate de 71 rehenes secuestrados por el MRTA
%\footnote{Ref. \url{http://peru.com/2012/04/22/actualidad/cronicas-y-entrevistas/fg-noticia-60015}.}
. Esto demuestra que la correcta aplicación del planeamiento estratégico, dentro del campo militar, puede salvar muchas vidas, tanto de soldados como civiles.

Por otro lado, en el Comando Operacional, el planeamiento de operaciones militares es un proceso largo y complejo. En el que se requiere la intervención de varios expertos y la recolección de información precisa. Sin embargo, esto muchas veces no es posible, debido a que los expertos deben rotar anualmente de diferentes departamentos. Asimismo, algunas veces la información utilizada en el planeamiento está desactualizada, impidiendo que los cálculos sean correctos y precisos.
Para solucionar este problema, se propone la construcción de un sistema experto, que brinde soporte en el planeamiento estratégico militar, para un caso determinado. Este software, contará con el conocimiento base del Plan Verde\footnote{Nombre clave de un plan asociado al experto militar que asesora la presente investigación.} y de la Doctrina del Estado Mayor. Además, el sistema permitirá la autenticación del usuario y mostrará las opciones según su rol, que puede ser Administrador o Experto.

El sistema propuesto, determinará los procedimientos militares a llevar a cabo en una determinada situación (escenario). Esto, tomando en cuenta variables como: el clima, el terreno, cantidad de soldados y armamento.
El sistema experto brindará soporte durante el análisis de una misión, haciendo que se reduzca la complejidad y el plan de la misma se realice con mayor rapidez. Además, permitirá el almacenamiento de diferentes escenarios de batalla con su respectivo análisis para posteriores consultas.
