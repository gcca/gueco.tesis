\chapter{Marco Teórico}


%% >>> : - . - . - . - . - o - . - . - . - . - :

\section{Fundamentos de las operaciones tácticas}

El contexto que un estratega del ejército tiene que analizar al momento de tomar una decisión es dada por el escenario, el cual puede ser un conflicto armado, un conflicto civil, un desastre natural, un rescate, entre otros. Para cada escenario se toman en cuenta factores que afectan a las acciones que se lleven a cabo, siendo los principales: el terreno, el clima y la capacidad armada de ambos bandos. Cada factor, identificado y definido con las características que posee, brinda información valiosa al estratega, quien se vale de dicho conocimiento del entorno para aplicar su capacidad en la toma de decisiones de las acciones que deben ser ejecutadas para actuar sobre el problema y reducir el daño que este puede causar.
El soporte tecnológico que brinda asistencia en temas relacionados a la aplicación de un conocimiento concreto, que partiendo de unas premisas definidas y a través de un razonamiento formalizado llega a conclusiones.

En una situación de conflicto, los Comandantes aplican ciertos fundamentos o normas en el planeamiento y ejecución de operaciones tácticas. Esto les permite optimizar el desempeño de sus Fuerzas Militares.\footnote{Ref. \citealt{Asesores12}.}

El éxito de un combate no depende únicamente de la tecnología armamentística, sino también de los recursos humanos con los que se cuenta. Por lo que el adiestramiento y la moral de los soldados es un factor que podría definir el resultado de la batalla.\footnote{Ref. \citealt{Asesores12}.}

Por otro lado, las condiciones del lugar donde se desarrolla el enfrentamiento ofrecen oportunidades para emplear eficazmente las armas, líneas de comunicación, y direcciones de acercamiento que facilitan la aplicación del poder de combate. Por eso, los Comandantes deben aprovechar cualquier ventaja que ofrezcan el terreno, las condiciones meteorológicas, el espacio aéreo y el tiempo\footnote{Ref. \citealt{Asesores12}.}. Para esto, se deben llevar a cabo acciones de reconocimiento que, además de brindar información sobre factores del medio ambiente, proveen datos del enemigo; como ubicación, armamento y cantidad de soldados.\footnote{Ref. \citealt{Asesores12}.}


%% >>> : - . - . - . - . - o - . - . - . - . - :

\section{Sistemas basados en conocimiento}

Los sistemas basados en conocimiento son aplicaciones que buscan imitar las actividades de un experto humano en un ámbito definido de su actividad. No pretenden recrear un pensamiento humano sino la experiencia a través de los conocimientos de varios expertos que incorporan su experiencia al sistema. Son el resultado de un largo proceso de investigación en el área de la Inteligencia Artificial en la década de los 70, cuando se estudió la capacidad de un programa para resolver problemas, en los que no se encuentran esquemas lógicos, sino en el conocimiento que estos representan\footnote{Cfr. \citealt{Labrana12}.}.

Durante la década de los 80 se desarrollaron métodos que servían para modelar sistemas basados en el conocimiento, basados en los trabajos de conocimiento de Allen Newell, investigador en informática y sicología cognitiva. Estos métodos se diferencian por la estructura que proponen para representar y analizar el conocimiento, el nivel de detalle de la tarea y su relación con el código ejecutable, coincidiendo en la elaboración del modelo conceptual capaz de describir el conocimiento del sistema.

Inicialmente el desarrollo de sistemas de conocimiento estaba dirigido por el paradigma de desarrollo por prototipos y de representación del conocimiento a través de reglas de producción, en muchos casos usando hardware y software de propósito especial, como CLISP\footnote{Ref. \url{http://www.clisp.org/summary.html}} o PROLOG\footnote{Ref. \url{http://www.visual-prolog.com}}. Se buscaba elaborar un estándar para ingeniería del conocimiento y sistemas de conocimiento con el cual se pudieran construir de manera que fueran sistemas industriales de calidad.


%% >>> : - . - . - . - . - o - . - . - . - . - :

\section{CommonKADS}

Como se detalla en \citealt{Henao01}, CommonKADS es la metodología que permite detectar las oportunidades en las organizaciones donde se desarrollan y aplican recursos de conocimiento; así, proporciona herramientas para la gestión del conocimiento de la organización. También da métodos para llevar a cabo un análisis detallado de las tareas y los procesos en los que se aplica el conocimiento. Otra cualidad es que toma en cuenta tanto la experiencia como las estructuras de información que existen en la organización ya que refleja la influencia de paradigmas como el análisis y el diseño estructurado, la teoría de las organizaciones, la orientación por objetos, la gerencia de calidad y la reingeniería de procesos. También facilita el desarrollo del sistema,donde permite obtener las especificaciones y los requerimientos de un problema y su solución.
Esta metodología fue conocida inicialmente como KADS (Knowledge Acquisition Designer System) porque se buscaba un método para la adquisición del conocimiento en el proceso de construcción de un sistema basado en conocimiento. Después se amplió el proyecto a la construcción de una metodología completa debido a los buenos resultados que obtuvo y fue así como paso a ser CommonKADS, la cual empieza desde el análisis de la organización hasta la gestión del proyecto.
CommonKADS se basa en un ciclo de vida donde el desarrollo se divide en una serie de fases con un orden predeterminado. En cada fase debe llevarse a cabo actividades distintas y la final de cada fase se obtiene documentos, informes, diseños o programas para poder pasar a la otra fase. Las fases principales de esta metodología son:

\begin{description}
\item[Análisis] enfocado a entender el problema. Los documentos que se deben obtener de esta fase son: documento del proyecto, de los requerimientos, del modelo conceptual y de viabilidad.
\item[Diseño] para una descripción física en la que se especifican los componentes del sistema, también se hace una descripción de su comportamiento.
\item[Implementación del sistema] que considera tanto la integración del software como su adaptación en la organización.
%\item[Instalación]
\item[Uso] para entender el manejo del sistema y los resultados que proporciona.
\end{description}

Los modelos de CommonKADS son estructuras de requerimientos del sistema de conocimiento, que partiendo de diferentes aspectos sirven para describir el conocimiento de la solución de problemas. Siendo estos modelos:

\begin{description}

\item[Modelo de la Organización (OM)] Analiza la organización.
  \begin{description}
  \item[OM-1] Identificación de los problemas y oportunidades en la organización, orientados al conocimiento.
  \item[OM-2] Descripción de los aspectos de la organización que tienen impacto o están afectados.
  \item[OM-3] Descripción del proceso desde el punto de vista de las tareas que lo conforman.
  \item[OM-4] Descripción del componente de conocimiento del modelo de la organización y sus principales características.
  \item[OM-5] Documento de viabilidad de la decisión.
  \end{description}

\item[Modelo de Tarea (TM)] Describe a un nivel general las tareas que son realizadas en el entorno de la organización, proporcionando la distribución de tareas entre agentes.
  \begin{description}
  \item[TM-1] Descripción de las tareas dentro del proceso objetivo.
  \item[TM-2] Especificación del conocimiento empleado por una tarea.
  \end{description}

\item[Modelo de Agente (AM)] Un agente es un ejecutor de una tarea que puede ser humano, software o cualquier otra entidad capaz de realizarla. En este modelo se describen las competencias, características, autoridad y restricciones para actuar que poseen los agentes.
  \begin{description}
  \item[AM-1] Formulario agente.
  \end{description}
\item[Modelo de Comunicaciones (CM)] Detalla el intercambio de información entre los diferentes agentes involucrados en la ejecución de las tareas descritas en el modelo de tarea. Los formularios de estos modelos son:
  \begin{description}
  \item[CM-1] Especificación de las transacciones que participan en el diálogo entre dos agentes en el modelo de comunicaciones.
  \item[CM-2] Especificación de los mensajes y la información que forman una transacción individual dentro del modelo de comunicaciones.
  \end{description}

\item[Modelo del Conocimiento (EM)] Este es el modelo principal de la metodología CommonKADS y modela el conocimiento de resolución de problemas empleado por un agente para realizar una tarea. El modelo de la experiencia distingue entre el conocimiento de la aplicación y el conocimiento de resolución del problema. El conocimiento de la aplicación se divide en tres subniveles: nivel del dominio (conocimiento declarativo sobre el dominio), nivel de inferencia (estructuras genéricas de inferencia) y nivel de tarea (orden de las inferencias).
\end{description}

Estos modelos están clasificados en tres niveles, los cuales segmentan el proceso de elaboración del sistema experto en fases que cubren la descripción del contexto en el que se desarrolla, el conocimiento involucrado con aquello que se requiere modelar y la parte técnica referente al desarrollo del sistema. Dichos niveles son:
\begin{description}
\item[Nivel del entorno] Relaciona la información del entorno del sistema de conocimientos. Implica tener conocimiento del contexto de la organización, su ambiente y de los factores críticos de éxito que corresponden al sistema. En este nivel se encuentran los modelos Organizacional, de Tareas y de Agentes.

\item[Nivel conceptos] Especifica lo que se quiere modelar. Es necesario tener modelos que presenten la descripción conceptual del modelo aplicado a una tarea y los datos que son manejados y entregados por un sistema. En este nivel se tiene el modelo de conocimientos y el de comunicación.

\item[Nivel de componentes] Identifica los aspectos técnico de programación y de construcción en el ordenador. En este nivel está el modelo de diseño.
\end{description}

Los modelos de experiencia y agentes proporcionan los requisitos de entrada que guiarán la implementación del sistema a través del modelo de diseño.
CommonKADS también plantea una serie de consideraciones para la gestión del proyecto de conocimiento formada por cuatro actividades:

\begin{description}
\item[Revisión] Revisa el estado actual del proyecto y se definen los principales objetivos para el siguiente ciclo. Este es el primer estado de cada ciclo.
\item[Riesgo] Identifica los obstáculos que se pueden presentar en el desarrollo del proyecto y que pueden impiden que se cumpla con los objetivos definitivos.
  \begin{description}
  \item[PM-1] Identificación y valoración de los riesgos del proyecto.
  \end{description}
\item[Plan] Se hace un plan detallado para el siguiente ciclo. Cuenta con el formulario:
  \begin{description}
  \item[PM-2] Como describir el estado del modelo como un objetivo a ser alcanzado en el proyecto.
  \end{description}
\item[Seguimiento] Registra los cambios o resultados obtenidos.
\end{description}

CommonKADS es importante dentro de las organizaciones porque ofrece un marco para la especificación del conocimiento independiente de la implementación, combinando un conjunto de modelos de conocimiento reutilizable. Además propone un ciclo de vida en donde se indican las fases, las actividades y los productos más relevantes para un proyecto de desarrollo de un sistema basado en conocimientos.

Una de las principales cualidades de CommonKADS es el planteamiento del desarrollo de modelos que reflejan diferentes vistas del proyecto. Entre ellos se resalta el modelo de conocimiento en el que las partes que lo conforman son independientes del dominio.
Los modelos que propone la metodología permiten reflejar diferentes visiones fundamentales en el sistema, desde el punto de vista de la empresa hasta la forma como éste se diseña. Esta metodología es muy amplia, todos los aspectos que se necesitan para llevar a cabo un buen proyecto de desarrollo de un sistema basado en conocimientos, desde el estudio del problema hasta la implantación del software y su gestión.






%%%%%%%%%%%%%%%%%%%%%%%%%%%%%%%%%%%%%%
% FALTA
%
%%%%%%%%%%%%%%%%%%%%%%%%%%%%%%%%%%%%%%
