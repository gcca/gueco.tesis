%% ACRÓNIMOS
%% ---------
\newacronym{picb}{PICB}{Preparación de Inteligencia del Campo de Batalla}
\newacronym{clips}{CLIPS}{C Language Integrated Production System}

\newacronym{em}{EM}{Estado mayor}
\newacronym{cicte}{CICTE}{Centro de Investigación de Ciencia y Tecnología del Ejército}
\newacronym{qa}{QA}{Quality Assurance}
\newacronym{sbc}{SBC}{Sistema basado en conocimiento}

%% GLOSARIO
%% --------
\newglossaryentry{batalla} {
  name={batalla},
  description={Confrontación bélica entre dos fuerzas de efectivos
               muy importantes. Normalmente es realizada por unidades
               capaces de actuar en más de una dirección.}
}

\newglossaryentry{escenario} {
  name={escenario},
  description={Lugar geográfico descrito en función del terreno,
               clima y unidades de combate}
}

\newglossaryentry{operacion} {
  name={operación},
  plural={operaciones},
  description={Aplicación de los principios de planificación,
               organización y administración en el uso de los recursos
               y de la fuerza militar de las unidades de combate
               para conseguir metas u objetivos específicos}
}

\newglossaryentry{maniobra} {
  name={maniobra},
  description={Movimiento logístico de unidades de combate, como tropas,
               batallones, tanques, aviones, entre otros}
}

\newglossaryentry{Gueco} {
  name={Gueco},
  description={Sistema experto de soporte en el planeamiento estratégico militar}
}

\newglossaryentry{campo} {
  name={campo},
  description={Ver \gls{escenario}}
}

\newglossaryentry{estado-mayor} {
  name={estado mayor},
  description={Conjunto de comandantes que toman las decisiones.
               en el proceso de la glspicb}
}

\newglossaryentry{comandante} {
  name={comandante},
  description={Militar que ejerce el mando de una fuerza, cualquiera que sea
               la magnitud o naturaleza de esta. Término con que, usualmente
               en el Ejército se denomina al Teniente Coronel}
}

\newglossaryentry{CommonKADS} {
  name={CommonKADS},
  description={Estándar para el desarrollo de sistemas basados en conocimiento
               en la gestión del conocimiento}
}

\newglossaryentry{plan} {
  name={plan},
  description={Forma de acción, generalmente escrita, que prescribe
               un conjunto de medidas para alcanzar una finalidad terminada}
}

\newglossaryentry{teniente} {
  name={Teniente},
  description={Segundo grado de la jerarquía del Oficial en el Ejército}
}

\newglossaryentry{teniente-coronel} {
  name={Teniente Coronel},
  description={Quinto grado de la jerarquía del Oficial en el Ejército.
               También se le denomina Comandante}
}

%%% Local Variables:
%%% mode: latex
%%% TeX-master: "tesis"
%%% End:
