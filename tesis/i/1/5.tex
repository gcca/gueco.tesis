\section{Alcance}
El alcance.\todo{¿Acá debería estar el plan de trabajo?}

\section{Plan de trabajo}

\todo{Detallar que significa semana,etc}
Para el desarrollo de la tesis se planteó el plan de trabajo \ref{cuadro:plandetrabajo}, cubierto en dos ciclos académicos con 15 semanas de trabajo cada uno.

\begin{longtable}[H]{>{\vspace{4px}}p{.8\textwidth}|>{\hfill}p{.13\textwidth}}
  \textbf{Hitos} & \textbf{Semana} \\\toprule
  \endhead

  \multicolumn{2}{c}{\textbf{Metodología CommonKADS}}\\\midrule

  Captura de información & 4
  \\%\midrule

  Modelo de organización & 4
  \\%\midrule

  Modelo de tareas & 6
  \\%\midrule

  Modelo de agentes & 8
  \\%\midrule

  Modelo contextual & 8
  \\%\midrule

  Modelo de comunicaciones & 9
  \\%\midrule

  Modelo de conocimiento & 10
  \\%\midrule

  Modelo conceptual & 11
  \\\midrule

  \multicolumn{2}{c}{\textbf{Fase de
diseño}}\\\midrule

  Esquema de desarrollo & 12
  \\%\midrule

  Prototipos de interfaces gráficas de usuario & 14
  \\\midrule

  \multicolumn{2}{c}{\textbf{Metodología CommonKADS}}\\\midrule

  Modelo de artefactos & 14
  \\\midrule

  \multicolumn{2}{c}{\textbf{Fase de desarrollo}}\\\midrule

  Desarrollo de base de conocimiento & 18
  \\%\midrule

  Desarrollo de base de datos & 19
  \\%\midrule

  Manejo de usuarios & 20
  \\%\midrule

  Manejo de lista de escenarios & 23
  \\%\midrule

  Edición de escenario y mapa & 26
  \\%\midrule

  Integración con motor de inferencia & 28
  \\\bottomrule
  \caption{Plan de trabajo}
  \label{cuadro:plandetrabajo}
\end{longtable}

%%% Local Variables:
%%% mode: latex
%%% TeX-master: "../../../tesis"
%%% End:
