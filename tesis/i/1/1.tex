\section{Definición del problema}

Antes que una batalla se lleve a cabo, los comandantes deben planificar cómo es que sus tropas, utilizando los recursos disponibles, van a desenvolverse en el campo. Esto, con la finalidad de lograr un objetivo, minimizando pérdidas en tropas y equipo. Para lograr esto, se realiza el proceso de \gls{picb}, que basicamente está divido en dos grupos de actividades. El primero, consiste en la obtención de información relevante para la batalla, por ejemplo: el estado del terreno, el clima, las unidades disponibles, datos del enemigo, etc. El segundo, es donde se analiza la información, por uno o más comandantes, para la planificación de una estrategia. 

Al finalizar este proceso, se deja establecido en un documento las \glspl{operacion} y \glspl{maniobra} que deben ejecutarse durante la batalla. El éxito de la misma, depende directamente de la correcta interpretación y ejecución de este. Por ende, es crucial que no contenga ningún tipo error y esté libre de ambigüedades. 

Por otro lado, el conocimiento que implica el análisis es muy amplio, debido a que involucra tanto conceptos teóricos como empíricos. Esto ocasiona que, en situaciones complejas, los expertos estén propensos a omitir principios básicos o
enfrenten situaciones en las que carecen de experiencia.


No obstante, la posibilidad de error está siempre presente por motivos como:\todo{creo que estas causas se centran en un problema que no tratamos de solucionar. Se refieren más a la sustitución del experto humano por el sistema}
\begin{itemize}
\item el especialista no pudo transmitir su conocimiento efectivamente,
\item error humano producido por el agotamiento,
\item mal cálculo debido a la complejidad del problema
\item u otros motivos, principalmente asociados a la condición humana.
\end{itemize}

A continuación se listan los principales problemas que influyen en el error producido durante el planeamiento estratégico.\todo{Este párrafo no me gusta.}

\begin{enumerate}
  \renewcommand{\labelitemi}{-}  % Cambia los 'items' por guiones

\item Respuesta lenta por parte del experto

  \begin{itemize}
  \item Los expertos humanos sufren de agotamiento físico, haciendo que se retrasen en el diseño de sus estrategias.
  \item Al tratar un escenario complejo el experto necesita más tiempo para el análisis.
  \end{itemize}

\item Ausencia del experto humano

  \begin{itemize}
  \item El experto puede estar realizando otra misión.
  \item El experto podría estar enfermo.
  \end{itemize}

\item Error del experto humano

  \begin{itemize}
  \item Error al realizar los cálculos, analizar la situación o emocionalmente no se encuentra apto para realizar su labor.
  \end{itemize}

\item Conocimiento incompleto

  \begin{itemize}
  \item Parte del conocimiento es obtenido empíricamente.
  \item El experto sólo domina un tipo de estrategia.
  \end{itemize}

\end{enumerate}

Por lo expuesto, el problema que trata la tesis es el de \emph{reducir el error que se produce durante el diseño y análisis de las \glspl{operacion} militares.}\todo{creo que no debemos enfocar la tesis por este lado porque tendríamos que tener la tasa de error actual y la posterior con el uso del sistema}

%%% Local Variables:
%%% mode: latex
%%% TeX-master: "../../../tesis"
%%% End:
