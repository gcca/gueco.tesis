\section{Beneficios y Justificación}

La justificación de esta tesis se centra en crear una herramienta útil para un experto militar que participa en la \gls{picb}. Dicha justificación se sostiene en los beneficios que se presentan a continuación:

\emph{\footnotesize Beneficios tangibles}
\begin{description}
\item[B1] Reducción de errores en el diseño de los calcos.
\item[B2] Reducción de costos, al no requerir cosas físicas como hojas de calco, marcadores, mapas impresos, entre otros.
\item[B3] Base de escenarios preparado para ser consultados.
\item[B4] Formalización centralizada\footnote{Es decir, contenida en un mismo lugar, en un mismo servidor.}  del conocimiento de diversos expertos.
\item[B5] Servir como herramienta del experto militar en la fase de planificación.
\item[B6] Ser un medio para documentar aspectos de la planificación militar.

\end{description}

\emph{\footnotesize Beneficios intangibles}
\begin{description}
\item[B7] Disminución de la complejidad de las actividades de la \gls{picb}\todo{arreglar esto en todo lo que diga \emph{...planificación militar}}.
\end{description}

Los objetivos específicos descritos previamente cubren los beneficios mencionados, los mismos en los que se encuentra fundamentada la utilidad y proceso de elaboración de esta tesis.


%%% Local Variables:
%%% mode: latex
%%% TeX-master: "../../../tesis"
%%% End:
