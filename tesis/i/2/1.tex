\section{Planeamiento militar} \todo{Mejor tratar del PICB}
... El título es genérico

\section{Preparación de Inteligencia del Campo de Batalla (PICB)}
El diseño, análisis y ejecución de las operaciones militares consta de cinco fases:
\begin{enumerate}
\item Recepción de la misión.
\item Análisis de la misión.
\item Desarrollo del conocimiento de la operación.
\item Desarrollo de los planes y órdenes.
\item Supervisión y control.
\end{enumerate}

Durante el desarrollo de las fases se realizan un conjunto de actividades que tienen como objetivo documentar y analizar el \gls{campo} de batalla. Así, la \emph{recepción de la misión} comprende las actividades:\todo{Refinar toda la lista.}
\begin{enumerate}
\item Reorganización de las secciones del \gls{estado-mayor}.
\item Reunión del \gls{estado-mayor} (lugar, fecha, hora).
\item Reunir las herramientas.
\item Actualización de la información disponible.
\item Realizar evaluación inicial.
\item Guía inicial del Comandante (Diseño).
\item Formulación de la orden preparatoria.
\end{enumerate}

Para la presente tesis, es el \emph{análisis de la misión} la fase en la que se trabajará. Donde se desarrolla la \gls{picb} \todo{Mejorar texto.}

\begin{enumerate}
\item Intención.
\item Estado final deseado.
\item Criterios de finalización.
\item Condiciones de éxito.
\item Objetivos.
\item Hipótesis.
\end{enumerate}

Las fases del \gls{picb} son:

\begin{enumerate}
\item Determinación, evaluación y análisis del campo de batalla. (Zona de Acción o Sector Defensivo, Área de Influencia y Área de Interés).
\item Análisis del terreno  y de las  condiciones meteorológicas.
\item Análisis del enemigo. (Dispositivo, Composición, Fuerza, Actividades reveladoras recientes y actuales. Peculiaridades y deficiencias.)
\item Integración (Establecer capacidad del enemigo, calco de eventos y calco de apoyo a la decisión).
\end{enumerate}



%%% Local Variables:
%%% mode: latex
%%% TeX-master: "../../tesis"
%%% End:
