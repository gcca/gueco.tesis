\chapter{Descripción de la tesis}


%% >>> : - . - . - . - . - o - . - . - . - . - :

\section{Declaración del problema}

Durante el proceso de la \gls{picb} se requiere la presencia de expertos en el planeamiento estratégico militar, para el diseño, desarrollo y ejecución de las operaciones militares. Actualmente, se tiene enfocado los esfuerzos en la elaboración del \emph{Plan Verde}\footnote{Nombre clave de un documento de operaciones tácticas, usado como fuente de información durante el desarrollo de la tesis}, documento donde se establecen las operaciones tácticas a llevar a cabo en caso de una invasión de \emph{Negro}\footnote{Nombre clave de un hipotético país hostil.}. Debido a la importancia del contenido de dicho documento, la presencia de errores debe ser nula. Sin embargo, la posibilidad de error está siempre presente por diversos motivos, por ejemplo: si el especialista no pudo transmitir su conocimiento eficientemente o por un error humano producido por el agotamiento, un mal cálculo u otro motivo asociado a la condición humana.

\todo{Acá poner acerca del PICB de los calcos, cuadernos usados, etc.}
En el cuadro \ref{cuadro:problemasyposiblescausas} se listan los problemas que pueden ocasionar error en el planeamiento estratégico con sus posibles causas.

\begin{longtable}[H]{p{.4\textwidth}|p{.53\textwidth}}
  \textbf{Problemas identificados} &  \textbf{Causas}\\\toprule
  \endhead


  Respuesta lenta por parte del experto &
  \begin{minipage}{.53\textwidth}
    \begin{itemize}
    \item Los expertos humanos sufren de agotamiento físico, haciendo que se retrasen en el diseño de sus estrategias.
    \item Al analizar un escenario complejo el experto necesita más tiempo para el análisis.
    \end{itemize}
  \end{minipage}
  \\\midrule


  Ausencia del experto humano &
  \begin{minipage}{.53\textwidth}
    \begin{itemize}
    \item El experto puede estar realizando otra misión.
    \item El experto podría estar enfermo.
    \end{itemize}
  \end{minipage}
  \\\midrule


  Error del experto humano &
  \begin{minipage}{.53\textwidth}
    \begin{itemize}
    \item Error al realizar los cálculos, analizar la situación o emocionalmente no se encuentra apto para realizar su labor.
    \end{itemize}
  \end{minipage}
  \\\midrule


  Conocimiento incompleto &
  \begin{minipage}{.53\textwidth}
    \begin{itemize}
    \item Parte del conocimiento es obtenido empíricamente.
    \item El experto sólo domina un tipo de estrategia.
    \end{itemize}
  \end{minipage}
  \\\bottomrule
  %\caption[Problemas y posibles causas]{\mbox{Problemas y posibles causas.\newline Fuente: Elaboración propia.}}
  \caption{Problemas y posibles causas}
  \label{cuadro:problemasyposiblescausas}
\end{longtable}


%% >>> : - . - . - . - . - o - . - . - . - . - :

\section{Objetivo general de la tesis}

%Esta tesis tiene como objetivo principal desarrollar un sistema experto capaz de procesar la información acerca de una zona de conflicto, emitiendo una serie de consejos acerca de las posibles operaciones a realizar en una determinada situación; para lo cual usará los factores de terreno, clima, situación del ejército propio y ejército enemigo. Así mismo, la tesis se desarrollará en el marco del Plan Verde, es decir, todo lo que implique el desarrollo (contexto de organización, base de conocimiento, zonas de conflicto de análisis, reglas de inferencia y soporte software necesario) estarán en la linde del plan antes mencionado, buscando cubrir lo descrito en él.

El principal objetivo de la tesis es desarrollar un software que permita a un experto militar, que participa en la \gls{picb}, ser asistido en dicho proceso. Dicho software será un sistema de cómputo capaz de procesar información acerca de un escenario de conflicto, emitiendo una serie de consejos acerca de las maniobras posibles a ejecutar.

\todo{Revisar esta parte. Debe ir en otro cap}Esta tesis se desarrolla en el marco del Plan Tenacidad\footnote{Para el presente trabajo, el asesor militar proveyó el contexto de un caso práctico (planificación militar) denominado para esta tesis como \emph{Plan Tenacidad}}, es decir, todo lo que implique el desarrollo (contexto de organización, base de conocimiento, zonas de conflicto de análisis, reglas de inferencia y soporte software necesario) estarán en la linde del plan antes mencionado, buscando cubrir lo descrito en él.


%% >>> : - . - . - . - . - o - . - . - . - . - :

\section{Objetivos específicos de la tesis}

Los objetivos específicos de \emph{Gueco}, se fijaron con la intención de proveer una visión general acerca de las partes del problema.

\begin{description}
  \item[OE1] Formalización de conocimiento del experto militar en reglas para inferencia.
  \item[OE2] Ofrecer un mecanismo basado en mapas digitalizados.
  \item[OE3] Diseño de una base de conocimiento extensible.
  \item[OE4] El sistema experto ofrecerá resultados con una confianza mayor que 0,85.
  \item[OE5] Poder almacenar el conocimiento de diversos expertos militares.
\end{description}


%% >>> : - . - . - . - . - o - . - . - . - . - :

\section{Beneficios y Justificación}

La justificación de esta tesis se centra en crear una herramienta útil para un experto militar que participa en la \gls{picb}. Dicha justificación se sostiene en los beneficios que se presentan a continuación:

\emph{\footnotesize Beneficios tangibles}
\begin{description}
\item[B1] Reducción de errores en el diseño de los calcos.
\item[B2] Reducción de costos, al no requerir cosas físicas como hojas de calco, marcadores, mapas impresos, entre otros.
\item[B3] Base de escenarios preparado para ser consultados.
\item[B4] Formalización centralizada\footnote{Es decir, contenida en un mismo lugar, en un mismo servidor.}  del conocimiento de diversos expertos.
\item[B5] Servir como herramienta del experto militar en la fase de planificación.
\item[B6] Ser un medio para documentar aspectos de la planificación militar.

\end{description}

\emph{\footnotesize Beneficios intangibles}
\begin{description}
\item[B7] Disminución de la complejidad de las actividades de la \gls{picb}\todo{arreglar esto en todo lo que diga \emph{...planificación militar}}.
\end{description}

Los objetivos específicos descritos previamente cubren los beneficios mencionados, los mismos en los que se encuentra fundamentada la utilidad y proceso de elaboración de esta tesis.


%% >>> : - . - . - . - . - o - . - . - . - . - :

\section{Plan de trabajo}

\todo{Detallar que significa semana,etc}
Para el desarrollo de la tesis se planteó el plan de trabajo \ref{cuadro:plandetrabajo}, cubierto en dos ciclos académicos con 15 semanas de trabajo cada uno.

\begin{longtable}[H]{>{\vspace{4px}}p{.8\textwidth}|>{\hfill}p{.13\textwidth}}
  \textbf{Hitos} & \textbf{Semana} \\\toprule
  \endhead

  \multicolumn{2}{c}{\textbf{Metodología CommonKADS}}\\\midrule

  Captura de información & 4
  \\%\midrule

  Modelo de organización & 4
  \\%\midrule

  Modelo de tareas & 6
  \\%\midrule

  Modelo de agentes & 8
  \\%\midrule

  Modelo contextual & 8
  \\%\midrule

  Modelo de comunicaciones & 9
  \\%\midrule

  Modelo de conocimiento & 10
  \\%\midrule

  Modelo conceptual & 11
  \\\midrule

  \multicolumn{2}{c}{\textbf{Fase de
diseño}}\\\midrule

  Esquema de desarrollo & 12
  \\%\midrule

  Prototipos de interfaces gráficas de usuario & 14
  \\\midrule

  \multicolumn{2}{c}{\textbf{Metodología CommonKADS}}\\\midrule

  Modelo de artefactos & 14
  \\\midrule

  \multicolumn{2}{c}{\textbf{Fase de desarrollo}}\\\midrule

  Desarrollo de base de conocimiento & 18
  \\%\midrule

  Desarrollo de base de datos & 19
  \\%\midrule

  Manejo de usuarios & 20
  \\%\midrule

  Manejo de lista de escenarios & 23
  \\%\midrule

  Edición de escenario y mapa & 26
  \\%\midrule

  Integración con motor de inferencia & 28
  \\\bottomrule
  \caption{Plan de trabajo}
  \label{cuadro:plandetrabajo}
\end{longtable}
